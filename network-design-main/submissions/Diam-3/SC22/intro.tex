\section{Introduction}
\label{sec:intro}

- Recent low-diameter networks such as Slim Fly pushed the margin in terms of
  reducing cost and power while keeping high performance and resilience.

- However, these designs have fundamental flaws: 
%
(1) They either have diameter two, and are thus fundamentally limited in their
scale (e.g., Slim Fly, Slim NoC).
%
(2) Or, they have diameter three, but are very far away from the Moore Bound,
thus they have suboptimal scales, which increases cost and power consumption,
and suffering from performance loss due to the higher diameter.
%
(3) Or, they have even higher diameters, and are also far away from the MB,
which means they have all the issues from point~(2), while suffering even more
from performance loss because the diameter and thus average path length as even
higher.

- We propose a new family of network topologies that tackle these issues, based
  on a concept of polarity.  These networks:
%
(1) Have diameter 3 and are either very close or not far from the MB (Delorme +
Kronecker Product graphs).
%
(2) Have diameter 4 BUT very low average path length, close to 3 (Del-4).
%
Using the above we secure networks with very high scales, more (?) performance
and more resilience and lower cost than modern Diam-3 and Diam-4 topologies
such as Dragonfly or Fat Tree.


\maciej{Do we add modularity as a central paper part? May be a good idea (my
definition!), BUT - only if we have a design that actually is modular}
