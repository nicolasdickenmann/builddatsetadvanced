% Various IFs
%
\newif\iftr     % Full technical report
\newif\ifall    % Various stuff that might be useful but for now we don't want to use it
\newif\ifconf   % Submission to a conf or journal, with space contraints
\newif\ifsq     % Squeeze space?
\newif\ifnonb   % Non blind submission
\newif\iftodos

\newif\ifsqCAP
\newif\ifsqVS
\newif\ifsqEN
\newif\ifsqTIT

\sqtrue
\sqCAPtrue
\sqENtrue
\sqVStrue
\sqTITtrue


\newcommand{\tr}[1]{\iftr #1 \fi}
\newcommand{\all}[1]{\ifall #1 \fi}
\newcommand{\cnf}[1]{\ifconf #1 \fi}



%%%%%%%%%%%%%%%%%%%%%%%%%%%%%%%%%%%%%%%%%%%%%%%%%%%%%%%%%%%%%%%
% Various packages for general stuff
%
\usepackage{etex}
\usepackage{balance}
%\usepackage[sc]{mathpazo}
\usepackage{epstopdf}
\usepackage{placeins}



%%%%%%%%%%%%%%%%%%%%%%%%%%%%%%%%%%%%%%%%%%%%%%%%%%%%%%%%%%%%%%%
% Various packages for citations 
%
\usepackage{cite}



%%%%%%%%%%%%%%%%%%%%%%%%%%%%%%%%%%%%%%%%%%%%%%%%%%%%%%%%%%%%%%%
% General visual stuff
%
\usepackage{graphicx}
\usepackage{float}
\usepackage{dblfloatfix}
\usepackage{multirow}
\usepackage{rotating}
\usepackage[switch]{lineno} % Line numbers
\usepackage{makecell}
\usepackage{tabulary}
\usepackage{parcolumns}
\usepackage{tikz}
\usetikzlibrary{tikzmark}

% Fix to make tikzpicture (with “remember picture” and “tikzmark”) work with SIG ACM template?
%
\usepackage{xpatch}
\expandafter\xpatchcmd
\csname pgfk@/tikz/every picture/.@cmd\endcsname
{\thepage}{\arabic{page}}{}{}

% Stuff for cool small background for text
%
\tikzstyle{comment} = [draw, fill=blue!70, text=white, text width=3cm, minimum height=1cm, rounded corners, align=left, font=\scriptsize]
\tikzstyle{background_alg} = [draw, fill=blue!20, opacity=0.4, inner sep=4pt, rounded corners=2pt]

\usetikzlibrary{shapes}
\usetikzlibrary{plotmarks}
\usetikzlibrary{calc, fit}

\let\labelindent\relax
\usepackage{enumitem}



%%%%%%%%%%%%%%%%%%%%%%%%%%%%%%%%%%%%%%%%%%%%%%%%%%%%%%%%%%%%%%%
% Math stuff 
%
\usepackage{amsthm}
\usepackage{amsmath,amssymb,amsfonts}
\usepackage{mathtools,mathrsfs}

\newtheorem{theorem}{Theorem}[section]
\newtheorem{conjecture}[theorem]{Conjecture}
\newtheorem{proposition}[theorem]{Proposition}
\newtheorem{lemma}[theorem]{Lemma}
\newtheorem{corollary}[theorem]{Corollary}
\newtheorem{example}[theorem]{Example}
\newtheorem{definition}[theorem]{Definition}

\newtheorem{defn}{Definition}
\newtheorem{thm}{Theorem}
\newtheorem{clm}{Claim}
\newtheorem{crl}{Corollary}
\newtheorem{lma}{Lemma}

\DeclarePairedDelimiter{\ceil}{\lceil}{\rceil}



%%%%%%%%%%%%%%%%%%%%%%%%%%%%%%%%%%%%%%%%%%%%%%%%%%%%%%%%%%%%%%%
% Font stuff
%
\usepackage{soul}
\usepackage{fontawesome}
\usepackage{pifont}
\usepackage{textcomp}
\usepackage{booktabs}
\usepackage{url}
%\usepackage[hyphens]{url}
\usepackage{pbox}
\usepackage[normalem]{ulem}
%\usepackage[hidelinks]{hyperref}
\usepackage[10pt]{moresize}
%\usepackage{newpxtext}
%\usepackage{newpxmath}

\usepackage{cleveref}
\usepackage[utf8]{inputenc}
\crefname{section}{§}{§§}
\Crefname{section}{§}{§§}

\newcommand{\macb}[1]{\textbf{{#1}}}
\newcommand{\macbs}[1]{\textbf{{#1}}}

\newcommand{\noAnswer}{\textcolor{black}{\faQuestionCircle}}




%%%%%%%%%%%%%%%%%%%%%%%%%%%%%%%%%%%%%%%%%%%%%%%%%%%%%%%%%%%%%%%
% Stuff for space squeezing
%

\ifsqCAP
%
\usepackage[font={scriptsize}]{caption}
\usepackage[font={scriptsize}]{subcaption}
% \usepackage[font={bf,sf,small}]{caption}
% \usepackage[font={bf,sf,small}]{subcaption}
%
\else
%
\usepackage[font={footnotesize}]{caption}
\usepackage[font={footnotesize}]{subcaption}
%
\fi

\newcommand{\vspaceSQ}[1]{\ifsqVS\vspace{#1}\fi}
\newcommand{\enlargeSQ}[1]{\ifsqEN\enlargethispage{\baselineskip}\fi}

\ifsqTIT
%
\newcommand{\subparagraph}{}
\usepackage[compact]{titlesec}
\titlespacing*{\section}{0pt}{6pt}{2pt}
\titlespacing*{\subsection}{0pt}{5pt}{1pt}
\titlespacing*{\subsubsection}{0pt}{5pt}{1pt}
%
% \setlength{\dbltextfloatsep}{9pt plus 2pt minus 2pt}
% \setlength{\floatsep}{5pt plus 2pt minus 2pt}
% \setlength{\textfloatsep}{5pt plus 2pt minus 2pt}
% \setlength{\intextsep}{5pt plus 2pt minus 2pt}
% \setlength{\belowcaptionskip}{0pt}
% \setlength{\abovecaptionskip}{2pt}
% 
% \setlength{\tabcolsep}{2.5pt}
% \renewcommand{\arraystretch}{0.9}
%
\fi



%%%%%%%%%%%%%%%%%%%%%%%%%%%%%%%%%%%%%%%%%%%%%%%%%%%%%%%%%%%%%%%
% Color stuff
%
\usepackage{xcolor}
\definecolor{darkgrey}{RGB}{70,70,70}
\definecolor{lightgrey}{RGB}{200,200,200}
\definecolor{lyellow}{RGB}{255,255,100}
\definecolor{llyellow}{RGB}{250,250,180}
\definecolor{lgreen}{RGB}{144,238,144}

\usepackage[customcolors]{hf-tikz}
\hfsetbordercolor{white}
\hfsetfillcolor{vlgray}

\definecolor{vlgray}{rgb}{0.77 0.77 0.77}
\definecolor{ablack}{rgb}{0.2 0.2 0.2}
\definecolor{vllgray}{rgb}{0.9 0.9 0.9}
\definecolor{bblue}{rgb}{0.7 0.7 0.99}

\usepackage{colortbl}



%%%%%%%%%%%%%%%%%%%%%%%%%%%%%%%%%%%%%%%%%%%%%%%%%%%%%%%%%%%%%%%
% Listings stuff
%
\usepackage{inconsolata}
\usepackage{listings}


\ifsq
%
\lstset{language=C++,
        escapechar=|,
        keepspaces=false,
        frame=tb,
        framexleftmargin=1.5em,
        basicstyle=\tt\ssmall,
        columns=fixed,
        otherkeywords={Input,Output,enddo,forall,bool,true,false, int64_t, MPI_Op, down to},
        tabsize=2,
        breaklines=true,
        captionpos=b,
        belowskip=-2.5em,
        aboveskip=-0.5em,
        numbers=left,
        xleftmargin=1.5em,
        keywordstyle=\bfseries\color{black!400!black},
        stringstyle=\color{orange},
        commentstyle=\color{gray},
        numberstyle=\ssmall,numbersep=3pt,mathescape}
%
\else
%
\lstset{language=C++,
        escapechar=|,
        keepspaces=false,
        frame=tb,
        framexleftmargin=1.5em,
        basicstyle=\tt\ssmall,
        columns=fixed,
        otherkeywords={Input,Output,enddo,forall,bool,true,false, int64_t, MPI_Op, down to},
        tabsize=2,
        breaklines=true,
        captionpos=b,
        belowskip=0.0em,
        aboveskip=0.0em,
        numbers=left,
        xleftmargin=1.5em,
        keywordstyle=\bfseries\color{black!400!black},
        stringstyle=\color{orange},
        commentstyle=\color{gray},
        numberstyle=\ssmall,numbersep=3pt,mathescape}
%
\fi

\renewcommand{\lstlistingname}{Algorithm}% Listing -> Algorithm



%%%%%%%%%%%%%%%%%%%%%%%%%%%%%%%%%%%%%%%%%%%%%%%%%%%%%%%%%%%%%%%
% Comments
%

\newcommand{\maciej}[1]{\textcolor{blue}{[Maciej: #1]}}
\newcommand{\m}[1]{\textcolor{blue}{[Maciej: #1]}}
\newcommand{\cesar}[1]{\textcolor{teal}{[Cesare: #1]}}
\newcommand{\htor}[1]{\textcolor{blue}{[Torsten: #1]}}
\newcommand{\peterem}[1]{\textcolor{blue}{[Peterem: #1]}}
\newcommand{\jakub}[1]{\textcolor{blue}{[Jakub: #1]}}
\newcommand{\armon}[1]{\textcolor{blue}{[Armon: #1]}}
\newcommand{\kacper}[1]{\textcolor{blue}{[Kacper: #1]}}
\newcommand{\zur}[1]{\textcolor{blue}{[Zur: #1]}}
\newcommand{\broken}[1]{\uwave{#1}}
\newcommand{\fix}[2]{\uwave{#1} \textcolor{cyan}{[ ==$\rangle$ #2]}}
\newcommand{\into}[2]{\uwave{#1} \textcolor{cyan}{[ --$\rangle$ #2]}}
\newcommand{\goal}[1]{\noindent\textcolor{red}{[Goal: #1]}\par}
\newcommand{\impr}[1]{\noindent\textcolor{red}{[Improve: #1]}}
\newcommand{\todo}[1]{\noindent\textcolor{red}{[TODO: #1]}}
\newcommand{\info}[1]{\noindent\textcolor{blue}{[INFO: #1]}}

\newcommand{\estart}[0]{\begin{center}\line(1,0){250}\end{center}\info{This
section is an extended version that is visible for debugging purposes. It
probably needs cleaning}}

\newcommand{\eend}[0]{\ \newline\info{End of an extended version}\newline\line(1,0){250}}

\DeclareRobustCommand{\hll}[1]{{\sethlcolor{llyellow}\hl{#1}}}
\DeclareRobustCommand{\hlg}[1]{{\sethlcolor{lgreen}\hl{#1}}}

\newcommand{\add}{\hspace{19em}\makebox[0pt]{\color{llyellow}\rule[-0.1ex]{39em}{2ex}}\hspace{-19em}}

\newcounter{highlight}
\newcommand{\highlight}[1]{%
\stepcounter{highlight}\tikzmarkin{\thehighlight}(0.05,-0.1)(0.08,0.23)#1\tikzmarkend{\thehighlight}%
}

\newcounter{Ahighlight}
\newcommand{\Ahighlight}[1]{%
\stepcounter{Ahighlight}\tikzmarkin{\theAhighlight}(0.05,-0.12)(0.08,0.25)#1\tikzmarkend{\theAhighlight}%
}



%%%%%%%%%%%%%%%%%%%%%%%%%%%%%%%%%%%%%%%%%%%%%%%%%%%%%%%%%%%%%%%
% Various tricks 
%

% Footnote without a marker (\blfootnote}. These packages are needed to avoid
% the white space in front.

\usepackage{scalerel,stackengine}
\stackMath
\newcommand\rwh[1]{%
\savestack{\tmpbox}{\stretchto{%
  \scaleto{%
        \scalerel*[\widthof{\ensuremath{#1}}]{\kern-.6pt\bigwedge\kern-.6pt}%
                  {\rule[-\textheight/2]{1ex}{\textheight}}%WIDTH-LIMITED BIG WEDGE
                              }{\textheight}% 
}{0.5ex}}%
\stackon[1pt]{#1}{\tmpbox}%
}

\usepackage[hang,flushmargin]{footmisc} 

\newcommand\blfootnote[1]{%
  \begingroup
  \renewcommand\thefootnote{}\footnote{#1}%
  \addtocounter{footnote}{-1}%
  \endgroup
}

%% Fixed/scaling delimiter examples (see mathtools documentation)
\DeclarePairedDelimiter\abs{\lvert}{\rvert}
\DeclarePairedDelimiter\norm{\lVert}{\rVert}

%% Use the alternative epsilon per default and define the old one as \oldepsilon
\let\oldepsilon\epsilon
\renewcommand{\epsilon}{\ensuremath\varepsilon}

%% Also set the alternate phi as default.
\let\oldphi\phi
\renewcommand{\phi}{\ensuremath{\varphi}}

%% Nice formatting for C++
\newcommand{\Rplus}{\protect\hspace{-.1em}\protect\raisebox{.35ex}{\smaller{\smaller\textbf{+}}}}
\newcommand{\CC}{\mbox{C\Rplus\Rplus}\xspace}

\if 0

%%%%%%%%%%%%%%%%%%%%%%%%%%%%%%%%%%%%%%%%%%%%%%%%%%%%%%%%%%%%%%%
% Obsolete, old, might be useful sometimes
%
%\usepackage[font={bf,sf,scriptsize}]{subfig}
%\captionsetup[subfigure]{font={footnotesize},captionskip=3pt}

%\usepackage[linesnumbered,ruled,vlined]{algorithm2e}
\usepackage[linesnumbered,ruled]{algorithm2e}
%\usepackage{algpseudocode}
\usepackage{multicol}
\SetKwComment{Comm}{$\triangleright$\ }{}
\SetAlFnt{\scriptsize}
\SetAlCapFnt{\scriptsize}
\SetAlCapNameFnt{\scriptsize}
\SetKwInOut{Input}{Input}
\SetKwInOut{Output}{Output}

% \SetAlFnt{\scriptsize\tt}
% \SetAlCapFnt{\scriptsize\sf}
% \SetAlCapNameFnt{\scriptsize\sf}

\makeatletter
\NewDocumentCommand{\LeftComment}{s m}{%
\Statex \IfBooleanF{#1}{\hspace*{\ALG@thistlm}}\(\triangleright\) #2}
\makeatother

\fi
