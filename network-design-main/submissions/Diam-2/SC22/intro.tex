\section{Introduction}
\label{sec:intro}

\begin{enumerate}
    \item Copackaging - technology, benefits, motivation 
    \item Impact on network design - design constraints, growth/modularity, all-to-all BW
    \item We restrict the discussion to diameter-2 topologies - higher ingestion
bandwidth for given router radix, lower latency, scalable topologies that can support
several thousands of nodes with few tens of ports. 
\end{enumerate}

Recent low-diameter networks such as Slim Fly pushed the margin in terms of
reducing cost and power while keeping high performance and resilience.
We explore it as a potential topology to build scalable copackaged networks.
However, its practical applicability is limited because:
\begin{enumerate}
    \item Few design points (degrees for which topologies exist)
    \item Lack of flexibility - need to rewire entire network to add few extra nodes
    \item Low injection bandwidth for systems with overprovisioned degrees (what to do with empty ports).
            \klcomment{We discussed this in the meeting with Prof. Toersten. Current solution
                    is to have Abas' Cayley graphs in the arsenal to provide one more design point
                    for given degree. Not sure if that is good enough.}
    \item Complex Layout - what consequences does this have?
\end{enumerate} 


We propose new diameter-2 topologies based on polarity quotient graphs\klcomment{ER graphs} and 
group product\klcomment{Abas graphs} that overcome these limitations:
\begin{enumerate}
    \item Cover all degrees. They complement Slim Fly 
        by enhancing the design space of scalable algebraic graphs.
    \item Are expandable - further racks can be added on empty ports without affecting performance.
    \item Simplified Layout
\end{enumerate}
The proposed graphs provide even better scalability than Slimfly (MMS graphs), 
asymptotically reaching the Moore bound for diameter-2.
{\color{red}Laura and Kelly: A sentence to mention the plethora of math work that exists and also mention the specific graphs we are looking at here. Or should this be in the Background section?}


 
