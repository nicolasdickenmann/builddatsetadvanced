\section{Topologies}

\maciej{Kartik, maybe focus on BCube for a start?}


\subsection{Slim Fly}
 
Slim Fly~\cite{besta2014slim} is a state-of-the-art cost-effective
topology for large computing centers that uses mathematical optimization to
minimize diameter $D$ for a given radix $k$ while maximizing size $N$.  SF's
low diameter ($D = 2$) ensures the lowest latency for many traffic patterns and
it reduces the number of required network resources (packets traverse fewer
routers and cables), lowering cost, static, and dynamic power consumption. 
%
SF is based on graphs approaching the Moore Bound (MB): The upper bound on the
number of vertices in a graph with a given $D$ and $k'$. This ensures full
global bandwidth and high resilience to link failures due to good expansion
properties.
%
Next, SF is group hierarchical. A group is not necessarily complete but all the
groups are connected to one another (with the same number of global links) and
form a complete network of groups.
%
We select SF because it is a state-of-the-art design based on optimization that
outperforms virtually all other targets in most metrics and represents
topologies with $D = 2$.

\noindent
\macb{Associated Parameters}
%
$N_r$ and $k'$ depend on a parameter $q$ that is a prime power with certain
properties (detailed in the original work~\cite{besta2014slim}). Some
flexibility is ensured by allowing changes to $p$ and with a large number of
suitable values of the parameter~$q$. We use the suggested value of $p = \left\lceil {k'}/{2}
\right\rceil$.

\subsection{Dragonfly}
 
Dragonfly~\cite{kim2008technology} is a group hierarchical network with $D
= 3$ and a layout that reduces the number of global wires.  Routers form
complete {groups}; groups are connected to one another to form a complete
network of groups with one link between any two groups.
%
DF comes with an intuitive design and represents deployed networks with $D =
3$.

\noindent
\macb{Associated Parameters}
%
Input is: the group size $a$, the number of channels from one
router to routers in other groups $h$, and concentration $p$. We use the
\emph{maximum capacity} DF (with the number of groups $g = ah+1$) that is
\emph{balanced}, i.e., the load on global links is balanced to avoid
bottlenecks ($a = 2p = 2h$).
In such a DF, a single parameter $p$ determines all others.

\subsection{Jellyfish}
 
Jellyfish~\cite{singla2012jellyfish} networks are random regular graphs
constructed by a simple greedy algorithm that adds randomly selected edges
until no additions as possible. The resulting construction has good expansion
properties~\cite{bondy1976graph}. Yet, all guarantees are probabilistic and
rare degenerate cases, although unlikely, do exist. Even if $D$ can be
arbitrarily high in degenerate cases, usually $D < 4$ with much lower $d$.
%
We select JF as it represents flexible topologies that use randomization and
offer very good performance properties.


\noindent
\macb{Associated Parameters}
%
JF is flexible. $N_r$ and $k'$ can be arbitrary; we use parameters matching
less flexible  topologies.
%
% Since $D$ and $d$ depend on $k'$ and $N_r$, arbitrarily low-diameter JF
% networks can be constructed by growing $k'$ asymptotically faster with $N$
% than other topologies do. This can lead to cost savings when very high radix
% routers are available. We do not consider such topologies.
% 
% In JF, $k'$ and $p$ need not be constant in the network, however we only
% consider \emph{homogeneous} configurations.
%
To compensate for the different amounts of hardware used in different
topologies, we include a Jellyfish network constructed from the same routers
for each topology; the performance differences observed between those networks
are due to the different hardware and need to be factored in when comparing
the deterministic topologies. 

\subsection{Xpander}

Xpander~\cite{valadarsky2015} networks resemble JF but have a
deterministic variant.  They are constructed by applying one or more so called
$\ell$-\emph{lifts} to a $k'$-clique  $G$.
%
The $\ell$-lift of $G$ consists of $\ell$ copies of $G$, where for each edge
$e$ in $G$, the copies of $e$ that connect vertices $s_1, \dots, s_\ell$ to
$t_1, \dots, t_\ell$, are replaced with a \emph{random matching} (can be
derandomized): $s_i$ is connected to $t_{\pi(i)}$ for a random
$\ell$-permutation $\pi$.
%
This construction yields a $k'$-regular graph with $N = \ell k'$ and good
expansion properties. The randomized $\ell$-lifts ensure good properties in the
expectation.
%
% We omit the deterministic variant as it is significantly more complex and
% only prevents rare degenerate cases.
% 
% To create an Xpander with given dimensions, different sequences of lifts can
% be used. The most basic solution is to use a single lift for some (large)
% value of $k$.  However, it is not clear if the guarantees apply to lifts for
% high $k$.  Therefore, we also consider networks that are obtained using only
% repeated 2-lifts, which are proven to preserve good
% properties~\cite{valadarsky2015}.
%
We select XP as it offers the advantages of JF in a deterministically
constructed topology.


\noindent
\macb{Associated Parameters}
%
We create XP with a single lift of arbitrary $\ell$. Such XP is
flexible although there are more constraints than in JF.  Thus, we cannot
create matching instances for each topology. We select $k' \in \{16, 32\}$ and
$\ell = k'$, which is comparable to diameter-2 topologies. We also consider
$\ell = 2$ with multiple lifts as this ensures good
properties~\cite{valadarsky2015}, but we do not notice any additional speedup.
We use $p = \frac{k'}2$, matching the diameter-2 topologies. 

\subsection{HyperX}
 
HyperX~\cite{ahn2009hyperx} is formed by arranging vertices in an
$L$-dimensional array and forming a clique along each 1-dimensional row.
Several topologies are special cases of HX, including complete graphs,
hypercubes (HCs)~\cite{bondy1976graph}, and Flattened Butterflies
(FBF)~\cite{kim2007flattened}. HX is a generic design that represents a wide
range of networks. 


\noindent
\macb{Associated Parameters}
%
An HX is defined by a 4-tuple $(L, S, K, p)$. $L$ is the number of dimensions
and $D=L$, $S$ and $K$ are $L$-dimensional vectors (they respectively denote
the array size in each dimension and the relative capacity of links along each
dimension). Networks with uniform $K$ and $S$ (for all dimensions) are called
\emph{regular}. We only use regular $(L, S, 1, \cdot)$ networks with $L \in
\{2,3\}$.
%
HX with $L=2$ is about a factor of two away from the MB ($k' \approx 2
\sqrt{N_r}$) resulting in more edges than other topologies. Thus, we include
higher-diameter variants with $k'$ similar to that of other networks.
%
Now, for full bisection bandwidth (BB), one should set $p = \frac{k'}{2D}$.
Yet, since HX already has the highest $k'$ and $N_r$ (for a fixed $N$) among
the considered topologies, we use a higher $p = \frac{k'}D$ as with the other
topologies to reduce the amount of used hardware. As we do not consider
worst-case bisections, we still expect HX to perform well.

\subsection{Fat Tree}
 
%\enlargethispage{\baselineskip}
%
Fat tree~\cite{leiserson1996cm5} is based on the Clos
network~\cite{clos1953study} with disjoint inputs and outputs and
unidirectional links. By ``folding'' inputs with outputs, a multistage fat tree
that connects any two ports with bidirectional links is constructed. We use
three-stage FTs with $D = 4$; fewer stages reduce scalability while more stages
lead to high $D$.
%
% The fat tree topology~\cite{leiserson1996cm5} is based on the Clos
% interconnection network, which originally has disjoint inputs and outputs and
% is built of unidirectional links. By “folding” the inputs back to the
% outputs, a hierarchical interconnection network that can connect any port to
% any other using bidirectional links can be constructed. We only consider
% three-stage folded Clos networks, which have a diameter of 4; with less
% stages, not enough endpoints can be connected, while more stages lead to
% overly large diameter.
%
FT represents designs that are in widespread use and feature excellent
performance properties such as full BB and non-blocking routing. 


\noindent
\macb{Associated Parameters}
%
A three-stage FT with full BB can be constructed from routers with uniform
radix $k$: It connects ${k^3}/4$ endpoints using five groups of ${k^2}/4$
routers. Two of these groups, ${k^2}/2$ routers, form an \emph{edge group}
with ${k}/2$ endpoints.  Another two groups form an \emph{aggregation layer}:
each of the edge groups forms a complete bipartite graph with one of the
aggregation groups using the remaining ${k}/2$ ports, which are called
\emph{upstream}. Finally, the remaining group is called the \emph{core}: each
of the two aggregation groups forms a fully connected bipartite graph with the
core, again using the remaining ${k}/2$ upstream ports. This also uses all $k$
ports of the core routers.
%
Now, for FT, it is not always possible to construct a matching JF as $N/N_r$
can be fractional. In this case, we select $p$ and $k'$ such that $k = p+k'$
and ${k'}/p \approx 4$, which potentially changes $N$. Note also that for FT,
$p$ is the number of endpoints per edge router, while in the other topologies,
all routers are edge routers.

