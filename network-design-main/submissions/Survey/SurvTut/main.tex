\documentclass[lettersize,journal]{IEEEtran}
%\usepackage{array}
%\usepackage{textcomp}
%\usepackage{stfloats}
%\usepackage{verbatim}
%\usepackage{cite}

\newif\ifsq
\newif\ifall
\newif\ifconf
\newif\iftr
\newif\iftrcol

\allfalse
\trtrue
\trcolfalse
\conffalse
\sqtrue

\usepackage{etex}

% \usepackage[font={sf}]{caption}
% \usepackage[font={sf}]{subcaption}

\usepackage[font={bf,sf,scriptsize}]{caption}
\usepackage[font={bf,sf,scriptsize}]{subcaption}

%\usepackage[font={bf,sf,scriptsize}]{subfig}
%\captionsetup[subfigure]{font={footnotesize},captionskip=3pt}

%\usepackage[sc]{mathpazo}

\usepackage{graphicx}
\usepackage{booktabs}
\usepackage{epstopdf}
\usepackage{float}
\usepackage{url}
\usepackage{placeins}
\usepackage{amsmath,amssymb,amsfonts}
\usepackage{mathtools,mathrsfs}
%\usepackage{cite}
\usepackage{multirow}
\usepackage{rotating}
\usepackage{pbox}
\usepackage[normalem]{ulem}
%\usepackage[hyphens]{url}
%\usepackage[hidelinks]{hyperref}

\usepackage{inconsolata}
\usepackage{listings}

% \newcommand{\subparagraph}{}
% \usepackage[compact]{titlesec}
% \titlespacing*{\section}{0pt}{6pt}{2pt}
% \titlespacing*{\subsection}{0pt}{5pt}{1pt}
% \titlespacing*{\subsubsection}{0pt}{5pt}{1pt}

\usepackage[hidelinks]{hyperref}
\usepackage{cleveref}
\usepackage[utf8]{inputenc}
\crefname{section}{§}{§§}
\Crefname{section}{§}{§§}

\usepackage[10pt]{moresize}

\usepackage{xcolor}
\definecolor{darkgrey}{RGB}{70,70,70}
\definecolor{lightgrey}{RGB}{200,200,200}

\let\labelindent\relax
\usepackage{enumitem}

\usepackage{makecell}

\ifsq
\newcommand{\subparagraph}{}
\usepackage[compact]{titlesec}
\titlespacing*{\section}{0pt}{6pt}{2pt}
\titlespacing*{\subsection}{0pt}{5pt}{1pt}
\titlespacing*{\subsubsection}{0pt}{5pt}{1pt}
\fi

% \setlength{\dbltextfloatsep}{9pt plus 2pt minus 2pt}
% \setlength{\floatsep}{5pt plus 2pt minus 2pt}
% \setlength{\textfloatsep}{5pt plus 2pt minus 2pt}
% \setlength{\intextsep}{5pt plus 2pt minus 2pt}
% \setlength{\belowcaptionskip}{0pt}
% \setlength{\abovecaptionskip}{2pt}

\setlength{\tabcolsep}{2.5pt}
%\renewcommand{\arraystretch}{0.9}

%\usepackage[hang,flushmargin]{footmisc} 

\lstset{language=C,
        escapechar=|,
        keepspaces=false,
        frame=tb,
        framexleftmargin=1.5em,
        basicstyle=\tt\ssmall,
        columns=fixed,
        %otherkeywords={enddo,forall,bool,true,false, int64_t, MPI_Op, in, parallel, function},
        otherkeywords={enddo,forall,bool,true,false, int64_t, MPI_Op, function, V, LOAD, STORE, BLEND, CMP, OR, AND, MIN, MAX, ADD, MUL },
        tabsize=2,
        breaklines=true,
        captionpos=b,
        belowskip=-2em,
        numbers=left,
        xleftmargin=1.5em,
        keywordstyle=\bfseries\color{black!400!black},
        stringstyle=\color{orange},
        commentstyle=\color{darkgrey},
        numberstyle=\ssmall,numbersep=3pt,mathescape}

\usepackage[linesnumbered,ruled,vlined]{algorithm2e}
\SetAlFnt{\scriptsize\tt}
\SetAlCapFnt{\scriptsize\sf}
\SetAlCapNameFnt{\scriptsize\sf}

\usepackage{algpseudocode}
\algblock{ParFor}{EndParFor}
% customising the new block
\algnewcommand\algorithmicparfor{\textbf{parfor}}
\algnewcommand\algorithmicpardo{\textbf{do}}
\algnewcommand\algorithmicendparfor{\textbf{end\ parfor}}
\algrenewtext{ParFor}[1]{\algorithmicparfor\ #1\ \algorithmicpardo}
\algrenewtext{EndParFor}{\algorithmicendparfor}

\newcommand{\jd}[1]{\textcolor{blue}{[Jens: #1]}}
\newcommand{\maciej}[1]{\textcolor{blue}{[Maciej: #1]}}
\newcommand{\marcel}[1]{\textcolor{blue}{[Marcel: #1]}}
\newcommand{\htor}[1]{\textcolor{blue}{[Torsten: #1]}}
\newcommand{\broken}[1]{\uwave{#1}}
\newcommand{\fix}[2]{\uwave{#1} \textcolor{cyan}{[ ==$\rangle$ #2]}}
\newcommand{\into}[2]{\uwave{#1} \textcolor{cyan}{[ --$\rangle$ #2]}}
\newcommand{\florian}[1]{\textcolor{purple}{[Florian: #1]}}
\newcommand{\goal}[1]{\noindent\textcolor{red}{[Goal: #1]}\par}
\newcommand{\impr}[1]{\noindent\textcolor{red}{[Improve: #1]}}
\newcommand{\todo}[1]{\noindent\textcolor{red}{[TODO: #1]}}
\newcommand{\nono}[1]{\textcolor{purple}{[Nono: #1]}}

\newcommand{\macb}[1]{\textbf{\textsf{#1}}}
\newcommand{\macbs}[1]{{\small\textbf{\textsf{#1}}}}

\newtheorem{defn}{Definition}
\newtheorem{thm}{Theorem}
\newtheorem{clm}{Claim}
\newtheorem{crl}{Corollary}
\newtheorem{lma}{Lemma}

%\renewcommand{\goal}[1]{}

% Footnote without a marker (\blfootnote}.
% These packages are needed to avoid the white space in front.

\usepackage[hang,flushmargin]{footmisc} 

\newcommand\blfootnote[1]{%
  \begingroup
  \renewcommand\thefootnote{}\footnote{#1}%
  \addtocounter{footnote}{-1}%
  \endgroup
}

% ?
\usepackage{bm}

% Really wide hat 

\usepackage{scalerel,stackengine}
\stackMath
\newcommand\rwh[1]{%
\savestack{\tmpbox}{\stretchto{%
  \scaleto{%
      \scalerel*[\widthof{\ensuremath{#1}}]{\kern-.6pt\bigwedge\kern-.6pt}%
          {\rule[-\textheight/2]{1ex}{\textheight}}%WIDTH-LIMITED BIG WEDGE
            }{\textheight}% 
}{0.5ex}}%
\stackon[1pt]{#1}{\tmpbox}%
}

%%%%%%%%%%%%%%%%%%%%% STUFF for the algorithmic comments 

\def\HiLiGA{\leavevmode\rlap{\hbox to \hsize{\color{black!10}\leaders\hrule height 1\baselineskip depth 1ex\hfill}}}
\def\HiLiGB{\leavevmode\rlap{\hbox to \hsize{\color{black!25}\leaders\hrule height 1\baselineskip depth 1ex\hfill}}}
\def\HiLiGC{\leavevmode\rlap{\hbox to \hsize{\color{black!40}\leaders\hrule height 1\baselineskip depth 1ex\hfill}}}
\def\HiLiGD{\leavevmode\rlap{\hbox to \hsize{\color{black!55}\leaders\hrule height 1\baselineskip depth 1ex\hfill}}}
\def\HiLiGE{\leavevmode\rlap{\hbox to \hsize{\color{black!70}\leaders\hrule height 1\baselineskip depth 1ex\hfill}}}
\def\HiLiGF{\leavevmode\rlap{\hbox to \hsize{\color{black!85}\leaders\hrule height 1\baselineskip depth 1ex\hfill}}}

\usepackage{algpseudocode}
\usepackage{tikz}
\usetikzlibrary{calc}
\usepackage{xcolor}
\makeatletter

% parfor for algorithmic:
%
% \usepackage{etoolbox}
% \newcommand{\algorithmicdoinparallel}{\textbf{do in parallel}}
% \makeatletter
% \AtBeginEnvironment{algorithmic}{%
%   \newcommand{\FORALLP}[2][default]{\ALC@it\algorithmicforall\ #2\ %
%       \algorithmicdoinparallel\ALC@com{#1}\begin{ALC@for}}%
% }
% \makeatother

% to change colors
\newcommand{\fillcol}{green!20}
\newcommand{\bordercol}{black}

% code from Andrew Stacey (with small adjustment to the border color)
% http://tex.stackexchange.com/questions/51582/background-coloring-with-overlay-specification-in-algorithm2e-beamer-package
\newcounter{jumping}

%%%%%%%%%%%%%%%%%%%%% END of the algorithmic comments 

\newcommand*\samethanks[1][\value{footnote}]{\footnotemark[#1]}

% FOR ABSTRACT:
%
%% Motivation/problem statement: Why do we care about the problem? What
%% practical, scientific, theoretical or artistic gap is your research
%% filling?

%% Methods/procedure/approach: What did you actually do to get your
%% results? (e.g. analyzed 3 novels, completed a series of 5 oil
%% paintings, interviewed 17 students)

%% Results/findings/product: As a result of completing the above
%% procedure, what did you learn/invent/create?

%% Conclusion/implications: What are the larger implications of your
%% findings, especially for the problem/gap identified in step 1?
%The Slim Fly topology uses fewer routers and has a 25\% lower construction
%cost than a Dragonfly network with a comparable number of endpoints. 
%


%\DeclareUnicodeCharacter{0301}{*************************************}

\usepackage{fontawesome}
\usepackage{pifont}
\usepackage{textcomp}

\usepackage{xcolor}
\usepackage{soul}
\usepackage{dblfloatfix}

\usepackage{units}

%\ifconf
%\usepackage[firstinits=true,bibencoding=utf8]{biblatex}
%\addbibresource{references.bib}
%\fi
%
%\ifconf
%\renewcommand*{\bibfont}{\sf\ssmall}
%\fi

\newcommand{\noAnswer}{\textcolor{lightgray}{\faQuestionCircle}}


\ifCLASSOPTIONcompsoc
  % IEEE Computer Society needs nocompress option
  % requires cite.sty v4.0 or later (November 2003)
%  \usepackage[nocompress]{cite}
\else
  % normal IEEE
%  \usepackage{cite}
\fi

% *** GRAPHICS RELATED PACKAGES ***
%
\ifCLASSINFOpdf
  % \usepackage[pdftex]{graphicx}
  % declare the path(s) where your graphic files are
  % \graphicspath{{../pdf/}{../jpeg/}}
  % and their extensions so you won't have to specify these with
  % every instance of \includegraphics
  % \DeclareGraphicsExtensions{.pdf,.jpeg,.png}
\else
  % or other class option (dvipsone, dvipdf, if not using dvips). graphicx
  % will default to the driver specified in the system graphics.cfg if no
  % driver is specified.
  % \usepackage[dvips]{graphicx}
  % declare the path(s) where your graphic files are
  % \graphicspath{{../eps/}}
  % and their extensions so you won't have to specify these with
  % every instance of \includegraphics
  % \DeclareGraphicsExtensions{.eps}
\fi
% graphicx was written by David Carlisle and Sebastian Rahtz. It is
% required if you want graphics, photos, etc. graphicx.sty is already
% installed on most LaTeX systems. The latest version and documentation
% can be obtained at: 
% http://www.ctan.org/pkg/graphicx
% Another good source of documentation is "Using Imported Graphics in
% LaTeX2e" by Keith Reckdahl which can be found at:
% http://www.ctan.org/pkg/epslatex
%
% latex, and pdflatex in dvi mode, support graphics in encapsulated
% postscript (.eps) format. pdflatex in pdf mode supports graphics
% in .pdf, .jpeg, .png and .mps (metapost) formats. Users should ensure
% that all non-photo figures use a vector format (.eps, .pdf, .mps) and
% not a bitmapped formats (.jpeg, .png). The IEEE frowns on bitmapped formats
% which can result in "jaggedy"/blurry rendering of lines and letters as
% well as large increases in file sizes.



% *** SUBFIGURE PACKAGES ***
%\ifCLASSOPTIONcompsoc
%  \usepackage[caption=false,font=footnotesize,labelfont=sf,textfont=sf]{subfig}
%\else
%  \usepackage[caption=false,font=footnotesize]{subfig}
%\fi
% subfig.sty, written by Steven Douglas Cochran, is the modern replacement
% for subfigure.sty, the latter of which is no longer maintained and is
% incompatible with some LaTeX packages including fixltx2e. However,
% subfig.sty requires and automatically loads Axel Sommerfeldt's caption.sty
% which will override IEEEtran.cls' handling of captions and this will result
% in non-IEEE style figure/table captions. To prevent this problem, be sure
% and invoke subfig.sty's "caption=false" package option (available since
% subfig.sty version 1.3, 2005/06/28) as this is will preserve IEEEtran.cls
% handling of captions.
% Note that the Computer Society format requires a sans serif font rather
% than the serif font used in traditional IEEE formatting and thus the need
% to invoke different subfig.sty package options depending on whether
% compsoc mode has been enabled.


\hyphenation{op-tical net-works semi-conduc-tor}

\usepackage{anyfontsize}

%\setcounter{secnumdepth}{3}

\begin{document}


\title{Network Topologies: Structure and Performance}

\author{Maciej Besta$^1$, Kartik Lakhotia$^2$, Torsten Hoefler$^1$, Fabrizio Petrini$^2$\vspace{0.5em}\\
$^1$Department of Computer Science, ETH Zurich; $^2$Intel Labs}

\maketitle
\IEEEpeerreviewmaketitle

%\IEEEtitleabstractindextext{%
\begin{abstract}
%
Abstract.  Abstract.  Abstract.  Abstract.  Abstract.  Abstract.  Abstract.
Abstract.  Abstract.  Abstract.  Abstract.  Abstract.  Abstract.  Abstract.
Abstract.  Abstract.  Abstract.  Abstract.  Abstract.  Abstract.  Abstract.
Abstract.  Abstract.  Abstract.  Abstract.  Abstract.  Abstract.  Abstract.
Abstract.  Abstract.  Abstract.  Abstract.  Abstract.  Abstract.  Abstract.
Abstract.  Abstract.  Abstract.  Abstract.  Abstract.  Abstract.  Abstract.
Abstract.  Abstract.  Abstract.  Abstract.  Abstract.  Abstract.  Abstract.
Abstract.  Abstract.  Abstract.  Abstract.  Abstract.  Abstract.  Abstract.
Abstract.  Abstract.  Abstract.  Abstract.  Abstract.  Abstract.  Abstract.
Abstract.  Abstract.  Abstract.  Abstract.  Abstract.  Abstract.  Abstract.
Abstract.  Abstract.  Abstract.  Abstract.  Abstract.  Abstract.  Abstract.
Abstract.  Abstract.  Abstract.  Abstract.  Abstract.  Abstract.  Abstract.
Abstract.  Abstract.  Abstract.  Abstract.  Abstract.  Abstract.  Abstract.
Abstract.  Abstract.  Abstract.  Abstract.  Abstract.  Abstract.  Abstract.
Abstract.  Abstract.  Abstract.  Abstract.  Abstract.  Abstract.  Abstract.
Abstract.  Abstract.  Abstract.  Abstract.  Abstract.  Abstract.  Abstract.
%
\end{abstract}


% \begin{IEEEkeywords}
% Routing, Multipath Routing, Switching, Multipath switching, Path
% Diversity, Network Architectures, High-Performance Networks, Data Center
% Networks, ECMP, Ethernet, TCP/IP
% \end{IEEEkeywords}
% }


%\IEEEdisplaynontitleabstractindextext


\section{Introduction}
\label{sec:intro}

\begin{enumerate}
    \item Copackaging - technology, benefits, motivation 
    \item Impact on network design - design constraints, growth/modularity, all-to-all BW
    \item We restrict the discussion to diameter-2 topologies - higher ingestion
bandwidth for given router radix, lower latency, scalable topologies that can support
several thousands of nodes with few tens of ports. 
\end{enumerate}

Recent low-diameter networks such as Slim Fly pushed the margin in terms of
reducing cost and power while keeping high performance and resilience.
We explore it as a potential topology to build scalable copackaged networks.
However, its practical applicability is limited because:
\begin{enumerate}
    \item Few design points (degrees for which topologies exist)
    \item Lack of flexibility - need to rewire entire network to add few extra nodes
    \item Low injection bandwidth for systems with overprovisioned degrees (what to do with empty ports).
            \klcomment{We discussed this in the meeting with Prof. Toersten. Current solution
                    is to have Abas' Cayley graphs in the arsenal to provide one more design point
                    for given degree. Not sure if that is good enough.}
    \item Complex Layout - what consequences does this have?
\end{enumerate} 


We propose new diameter-2 topologies based on polarity quotient graphs\klcomment{ER graphs} and 
group product\klcomment{Abas graphs} that overcome these limitations:
\begin{enumerate}
    \item Cover all degrees. They complement Slim Fly 
        by enhancing the design space of scalable algebraic graphs.
    \item Are expandable - further racks can be added on empty ports without affecting performance.
    \item Simplified Layout
\end{enumerate}
The proposed graphs provide even better scalability than Slimfly (MMS graphs), 
asymptotically reaching the Moore bound for diameter-2.
{\color{red}Laura and Kelly: A sentence to mention the plethora of math work that exists and also mention the specific graphs we are looking at here. Or should this be in the Background section?}


 

\section{Fundamental Concepts \& Background}



We first introduce the notation and basic concepts. The most important used
symbols are summarized in Table~\ref{tab:symbols}. 
%
% An extended and more formal description of considered topologies, used parameters,
% and tested traffic patterns can be
% found in a technical report.

\begin{table}[h]
%
%\vspace{0.25em}
\sffamily
\centering
%\scriptsize
%\ssmall
\footnotesize
\setlength{\tabcolsep}{2pt}
%\renewcommand{\arraystretch}{0.6}
\begin{tabular}{@{}lll@{}}
\toprule
%
%\multirow{6}{*}{\begin{turn}{90}\shortstack{\textbf{Network structure }\\\textbf{(\cref{sec:background_topos}--\cref{sec:eval})}}\end{turn}} 
%
\iftr
\multirow{6}{*}{\begin{turn}{90}\shortstack{\textbf{Network }\\\textbf{structure}}\end{turn}} 
\fi
\ifconf
\multirow{6}{*}{\begin{turn}{90}\shortstack{\textbf{Network }\\\textbf{structure}}\end{turn}} 
\fi
                   & $V, E$ & Sets of vertices/edges (routers/links, $V=\{0,\dots,N_r-1\}$).\\
%                   & $E$ & Set of edges (links, $E \subseteq {V\choose 2}$)\\
%                   & $\mathcal{S}, \mathcal{E}$ & The set of all the routers and endpoints, $|\mathcal{S}|=N_r, |\mathcal{E}| = N$\\
%                   & $\mathcal{E}$ & The set of all the endpoints, $\{0,\dots,N-1\}$, $|\mathcal{E}|=N$.\\
                   & $N, N_r$& \#endpoints and \#routers in the network ($N_r = |V|$).\\
                   & $p$& \#endpoints attached to a router (\emph{concentration}).\\
                   & $k'$& \#channels to other routers (\emph{network radix}).\\
                   \iftr
                   & $k$&\emph{Router radix} ($k = k' + p$).\\
                   \fi
%                   & $D$&Network diameter\\
                   & $D, d$&Network diameter and the average path length.\\
%                   \iftr
%                  & & $q, a, h, \ell, L, S, K$: input parameters of various topologies \\ 
%\fi
%\cmidrule{3-3}
%                   & \multicolumn{2}{l}{Input parameters: $q$ (SF), $a,h$ (DF), $\ell$ (XP), $L,S,K$ (HX).} \\
%
\midrule
%
\iftr
\multirow{6}{*}{\begin{turn}{90}\shortstack{\textbf{Diversity of}\\\textbf{paths (\cref{sec:paths})}}\end{turn}} 
\fi
\ifconf
\multirow{6}{*}{\begin{turn}{90}\shortstack{\textbf{Diversity}\\\textbf{paths (\cref{sec:paths})}}\end{turn}} 
\fi
                   & $x \in V$ & Different routers used in~\cref{sec:paths} ($x \in \{s,t,a,b,c,d\}$).\\
%                   & $A,B,C,D$ & Distinct routers chosen uniformly at random from $V$.\\
\iftr
                   & $X \subset V$ & Different router sets used in~\cref{sec:paths} ($X \in \{A,B\}$).\\
\fi
%                   & $h^i(\cdot)$ & $i$-step transitive hull (used to define router neighborhoods).\\
                   & $c_l(A,B)$ & \emph{Count of (at most $l$-hop) disjoint paths} between router sets $A$, $B$.\\
                   & $c_\text{min}(s,t)$ & \emph{Diversity of minimal paths} between routers $s$ and $t$.\\
                   & $l_\text{min}(s,t)$ & \emph{Lengths of minimal paths} between routers $s$ and $t$.\\
                   & $I_{ac,bd}$ & \emph{Path interference} between pairs of routers $a,b$ and $c,d$.\\
\midrule
%
\multirow{3}{*}{\begin{turn}{90}\shortstack{\textbf{Layers}\\\textbf{(\cref{sec:routing})}}\end{turn}} 
%                   & $n$ & Total number of layers, fraction of edges in use \\
                   & $n$ & The total number of layers in FatPaths routing.\\
                   & $\sigma_i$ & A layer, defined by its forwarding function, $i\in \{1,\dots,n\}$.\\
                   & $\rho$ & Fraction of edges used in routing.\\

%\midrule
%
%\multirow{4}{*}{\begin{turn}{90}\shortstack{\textbf{Perf.}\\\textbf{(\cref{sec:eval})}}\end{turn}} 
%                   & $\mathcal{L}$ & The set of all the layers (a layer is a subset of $\mathcal{E}$)\\
%                   & $\phi$ & A hashing function\\
%                   & $\sigma$ & A forwarding function \\
%                   & $\lambda, v$ & Flow arrival rate [flows/s], flow volume [bytes].\\
%                   & $v$ & Flow volume [Bytes] \\
   %                   & $\mu$, $\delta$ & Net throughput [bytes/s], latency [s].\\
%                   & $\pi_N$ & A permutation of $N$ entries \\
   %                   & $\eta$ & Flow Completion Time (FCT) distribution model.\\
%
                   \bottomrule
\end{tabular}
%\vspace{-1em}
\caption{\textmd{The \textbf{most important symbols} used in this work.}}
\label{tab:symbols}
\end{table}

\input{taxonomy.tex}
\section{Topologies}

\maciej{Kartik, maybe focus on BCube for a start?}


\subsection{Slim Fly}
 
Slim Fly~\cite{besta2014slim} is a state-of-the-art cost-effective
topology for large computing centers that uses mathematical optimization to
minimize diameter $D$ for a given radix $k$ while maximizing size $N$.  SF's
low diameter ($D = 2$) ensures the lowest latency for many traffic patterns and
it reduces the number of required network resources (packets traverse fewer
routers and cables), lowering cost, static, and dynamic power consumption. 
%
SF is based on graphs approaching the Moore Bound (MB): The upper bound on the
number of vertices in a graph with a given $D$ and $k'$. This ensures full
global bandwidth and high resilience to link failures due to good expansion
properties.
%
Next, SF is group hierarchical. A group is not necessarily complete but all the
groups are connected to one another (with the same number of global links) and
form a complete network of groups.
%
We select SF because it is a state-of-the-art design based on optimization that
outperforms virtually all other targets in most metrics and represents
topologies with $D = 2$.

\noindent
\macb{Associated Parameters}
%
$N_r$ and $k'$ depend on a parameter $q$ that is a prime power with certain
properties (detailed in the original work~\cite{besta2014slim}). Some
flexibility is ensured by allowing changes to $p$ and with a large number of
suitable values of the parameter~$q$. We use the suggested value of $p = \left\lceil {k'}/{2}
\right\rceil$.

\subsection{Dragonfly}
 
Dragonfly~\cite{kim2008technology} is a group hierarchical network with $D
= 3$ and a layout that reduces the number of global wires.  Routers form
complete {groups}; groups are connected to one another to form a complete
network of groups with one link between any two groups.
%
DF comes with an intuitive design and represents deployed networks with $D =
3$.

\noindent
\macb{Associated Parameters}
%
Input is: the group size $a$, the number of channels from one
router to routers in other groups $h$, and concentration $p$. We use the
\emph{maximum capacity} DF (with the number of groups $g = ah+1$) that is
\emph{balanced}, i.e., the load on global links is balanced to avoid
bottlenecks ($a = 2p = 2h$).
In such a DF, a single parameter $p$ determines all others.

\subsection{Jellyfish}
 
Jellyfish~\cite{singla2012jellyfish} networks are random regular graphs
constructed by a simple greedy algorithm that adds randomly selected edges
until no additions as possible. The resulting construction has good expansion
properties~\cite{bondy1976graph}. Yet, all guarantees are probabilistic and
rare degenerate cases, although unlikely, do exist. Even if $D$ can be
arbitrarily high in degenerate cases, usually $D < 4$ with much lower $d$.
%
We select JF as it represents flexible topologies that use randomization and
offer very good performance properties.


\noindent
\macb{Associated Parameters}
%
JF is flexible. $N_r$ and $k'$ can be arbitrary; we use parameters matching
less flexible  topologies.
%
% Since $D$ and $d$ depend on $k'$ and $N_r$, arbitrarily low-diameter JF
% networks can be constructed by growing $k'$ asymptotically faster with $N$
% than other topologies do. This can lead to cost savings when very high radix
% routers are available. We do not consider such topologies.
% 
% In JF, $k'$ and $p$ need not be constant in the network, however we only
% consider \emph{homogeneous} configurations.
%
To compensate for the different amounts of hardware used in different
topologies, we include a Jellyfish network constructed from the same routers
for each topology; the performance differences observed between those networks
are due to the different hardware and need to be factored in when comparing
the deterministic topologies. 

\subsection{Xpander}

Xpander~\cite{valadarsky2015} networks resemble JF but have a
deterministic variant.  They are constructed by applying one or more so called
$\ell$-\emph{lifts} to a $k'$-clique  $G$.
%
The $\ell$-lift of $G$ consists of $\ell$ copies of $G$, where for each edge
$e$ in $G$, the copies of $e$ that connect vertices $s_1, \dots, s_\ell$ to
$t_1, \dots, t_\ell$, are replaced with a \emph{random matching} (can be
derandomized): $s_i$ is connected to $t_{\pi(i)}$ for a random
$\ell$-permutation $\pi$.
%
This construction yields a $k'$-regular graph with $N = \ell k'$ and good
expansion properties. The randomized $\ell$-lifts ensure good properties in the
expectation.
%
% We omit the deterministic variant as it is significantly more complex and
% only prevents rare degenerate cases.
% 
% To create an Xpander with given dimensions, different sequences of lifts can
% be used. The most basic solution is to use a single lift for some (large)
% value of $k$.  However, it is not clear if the guarantees apply to lifts for
% high $k$.  Therefore, we also consider networks that are obtained using only
% repeated 2-lifts, which are proven to preserve good
% properties~\cite{valadarsky2015}.
%
We select XP as it offers the advantages of JF in a deterministically
constructed topology.


\noindent
\macb{Associated Parameters}
%
We create XP with a single lift of arbitrary $\ell$. Such XP is
flexible although there are more constraints than in JF.  Thus, we cannot
create matching instances for each topology. We select $k' \in \{16, 32\}$ and
$\ell = k'$, which is comparable to diameter-2 topologies. We also consider
$\ell = 2$ with multiple lifts as this ensures good
properties~\cite{valadarsky2015}, but we do not notice any additional speedup.
We use $p = \frac{k'}2$, matching the diameter-2 topologies. 

\subsection{HyperX}
 
HyperX~\cite{ahn2009hyperx} is formed by arranging vertices in an
$L$-dimensional array and forming a clique along each 1-dimensional row.
Several topologies are special cases of HX, including complete graphs,
hypercubes (HCs)~\cite{bondy1976graph}, and Flattened Butterflies
(FBF)~\cite{kim2007flattened}. HX is a generic design that represents a wide
range of networks. 


\noindent
\macb{Associated Parameters}
%
An HX is defined by a 4-tuple $(L, S, K, p)$. $L$ is the number of dimensions
and $D=L$, $S$ and $K$ are $L$-dimensional vectors (they respectively denote
the array size in each dimension and the relative capacity of links along each
dimension). Networks with uniform $K$ and $S$ (for all dimensions) are called
\emph{regular}. We only use regular $(L, S, 1, \cdot)$ networks with $L \in
\{2,3\}$.
%
HX with $L=2$ is about a factor of two away from the MB ($k' \approx 2
\sqrt{N_r}$) resulting in more edges than other topologies. Thus, we include
higher-diameter variants with $k'$ similar to that of other networks.
%
Now, for full bisection bandwidth (BB), one should set $p = \frac{k'}{2D}$.
Yet, since HX already has the highest $k'$ and $N_r$ (for a fixed $N$) among
the considered topologies, we use a higher $p = \frac{k'}D$ as with the other
topologies to reduce the amount of used hardware. As we do not consider
worst-case bisections, we still expect HX to perform well.

\subsection{Fat Tree}
 
%\enlargethispage{\baselineskip}
%
Fat tree~\cite{leiserson1996cm5} is based on the Clos
network~\cite{clos1953study} with disjoint inputs and outputs and
unidirectional links. By ``folding'' inputs with outputs, a multistage fat tree
that connects any two ports with bidirectional links is constructed. We use
three-stage FTs with $D = 4$; fewer stages reduce scalability while more stages
lead to high $D$.
%
% The fat tree topology~\cite{leiserson1996cm5} is based on the Clos
% interconnection network, which originally has disjoint inputs and outputs and
% is built of unidirectional links. By “folding” the inputs back to the
% outputs, a hierarchical interconnection network that can connect any port to
% any other using bidirectional links can be constructed. We only consider
% three-stage folded Clos networks, which have a diameter of 4; with less
% stages, not enough endpoints can be connected, while more stages lead to
% overly large diameter.
%
FT represents designs that are in widespread use and feature excellent
performance properties such as full BB and non-blocking routing. 


\noindent
\macb{Associated Parameters}
%
A three-stage FT with full BB can be constructed from routers with uniform
radix $k$: It connects ${k^3}/4$ endpoints using five groups of ${k^2}/4$
routers. Two of these groups, ${k^2}/2$ routers, form an \emph{edge group}
with ${k}/2$ endpoints.  Another two groups form an \emph{aggregation layer}:
each of the edge groups forms a complete bipartite graph with one of the
aggregation groups using the remaining ${k}/2$ ports, which are called
\emph{upstream}. Finally, the remaining group is called the \emph{core}: each
of the two aggregation groups forms a fully connected bipartite graph with the
core, again using the remaining ${k}/2$ upstream ports. This also uses all $k$
ports of the core routers.
%
Now, for FT, it is not always possible to construct a matching JF as $N/N_r$
can be fractional. In this case, we select $p$ and $k'$ such that $k = p+k'$
and ${k'}/p \approx 4$, which potentially changes $N$. Note also that for FT,
$p$ is the number of endpoints per edge router, while in the other topologies,
all routers are edge routers.


\input{measures.tex}
\section{Conclusion}
\vspaceSQ{-0.25em}





\ifconf
\printbibliography
\fi



\iftr
  %\bibliographystyle{abbrv}
  \bibliographystyle{IEEEtran}
  \bibliography{references}
\fi



\ifconf
\fi
\iftr

\vspace{-3em}

\begin{IEEEbiographynophoto}{\scriptsize Maciej Besta}\scriptsize
%\begin{IEEEbiographynophoto}{Maciej Besta}
%
\end{IEEEbiographynophoto}

\vspace{-3em}
\begin{IEEEbiographynophoto}{\scriptsize Kartik Lakhotia}\scriptsize 
%\begin{IEEEbiographynophoto}{Jens Domke} 
%
\end{IEEEbiographynophoto}
%
\vspace{-3em}
\begin{IEEEbiographynophoto}{\scriptsize Torsten Hoefler}\scriptsize 
%\begin{IEEEbiographynophoto}{Marcel Schneider} 
%
\end{IEEEbiographynophoto}
%
\vspace{-3em}
\begin{IEEEbiographynophoto}{\scriptsize Fabrizio Petrini}\scriptsize 
%\begin{IEEEbiographynophoto}{Timo Schneider} 
%
\end{IEEEbiographynophoto}
%
\fi

\end{document}


